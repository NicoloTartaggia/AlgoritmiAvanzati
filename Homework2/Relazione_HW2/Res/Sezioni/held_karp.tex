\section{Algoritmi esatti}\label{prim}
\subsection{Held e Karp}
\begin{lstlisting}
	HELD_KARP(v,S)
		 if S = {v} then
		 	  return w[v,0]   
		 else if d[v,S] 6= NULL then
		 	return d[v,S]
		 else
		 	mindist = infinity
		 	minprec = NULL
		 	for all u in S \ {v} do 
		 		dist = HK-VISIT(u,S \ {v})
		 		if dist +w[u, v] < mindist then
		 			mindist = dist + w[u, v]
		 			minprec = u
		 	d[v,S] = mindist
		 	phi[v,S] = minprec  
		 	return mindist                      
\end{lstlisting}

L’algoritmo di Held e Karp è un algoritmo di programmazione dinamica per risolvere il problema del TSP proposto nel 1962. La programmazione dinamica è una tecnica che permette di risolvere problemi di ottimizzazione
combinando le soluzioni di sottoproblemi più semplici: l'algoritmo si occupa di risolvere ognuno dei sottoproblemi una volta sola, salvando poi il risultato in una tabella per poterlo riutilizzare in seguito senza doverlo risolvere nuovamente. \\
L'algoritmo si basa sul fatto che ogni sottocammino di un cammino minimo è a sua volta un cammino minimo e questa proprietà permette di risolvere il problema del TSP in modo ricorsivo.

\subsection{Strutture dati}
L'algoritmo si basa sulle seguenti strutture dati:
\begin{itemize}
	\item V: vettore di coppie in formato (i, [x, y]) dove i rappresenta l'indice progressivo nell'insieme \{0, .., n - 1\} e [x, y] corrisponde alle coordinate del punto;
	\item subsets: dizionario usato per enumerare i sottoinsiemi di V;
	\item d[v, S]: vettore a due dimensioni che contiene il peso del cammino minimo che parte da 0 e termina in v, visitando tutti i nodi in S;
	\item phi[v, S]: vettore a due dimensioni che contiene il predecessore di v nel cammino minimo.
\end{itemize}



\subsection{Implementazione}
La soluzione del problema è stata implementata nel seguente modo:
\begin{itemize}
	\item Viene ricavato l'indice corrispondente al sottoinsieme S presente all'interno di subsets;
	\item Se il tempo trascorso dall'inizio dell'esecuzione dell'algoritmo è superiore alla soglia la funzione ritorna il valore "None" come segnale per interrompere la computazione, altrimenti passa all'istruzione successiva;
	\item Se S ha un solo elemento ed esso coincide con v il valore ritornato sarà il peso dell'arco che ha come estremi 0 e v, altrimenti passa all'istruzione successiva;
	\item Se la soluzione al sottoproblema [v, S] è già presente in d il suo valore viene ritornato dall'algoritmo, altrimenti passa all'istruzione successiva;
	\item Nel caso ricorsivo inizialmente vengono inizializzati i valori di mindist e minprec, poi viene costruito il nuovo sottoinsieme S $\backslash$ \{v\} e se non è già presente in subsets, la sua enumerazione viene aggiunta incrementando il contatore. Per rendere univoco ogni sottoinsieme esso viene trasformato, tramite fa funzione encode, ottenendo così una stringa composta dagli indici dei vertici contenuti al suo interno separati da uno spazio (ad esempio l'insieme \{1, 2, 3\} verrà codificato nella stringa "1 2 3"). Per ogni vertice u presente nell'insieme S - \{v\} viene effettuata la chiamata ricorsiva a held\_karp(u, S - \{v\}): se il risultato di tale chiamata è None significa che il tempo limiti è stato superato e quindi il ciclo verrà interrotto restituendo la soluzione migliore finora trovata, altrimenti l'algoritmo cercherà la soluzione migliore al sottoproblema che, dopo aver aggiornato i vettori v e phi, sarà ritornata in output.
\end{itemize}


\subsection{Complessità}
La complessità dell'algoritmo è O($n^2$ · $2^n$), dove n indica il numero di vertici:
\begin{itemize}
	\item In ognuno dei casi base l'algoritmo ha complessità O(1);
	\item Nel caso ricorsivo il ciclo for itera su tutti i vertici nell'insieme S $\backslash$ \{v\} e quindi impiega tempo O(n). Il numero di chiamate ricorsive è limitato superiormente dal numero di elementi presenti in d[v, S] che è al più n · $2^n$, con v vertice del
	grafo e S un sottoinsieme di vertici. 
\end{itemize}
\pagebreak