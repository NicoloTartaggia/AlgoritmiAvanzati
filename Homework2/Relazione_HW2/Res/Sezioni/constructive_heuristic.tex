\section{Euristiche costruttive}\label{constructive_heuristic}

Il termine euristiche costruttive identifica un'ampia famiglia di euristiche che costruiscono la soluzione finale aggiungendo un vertice alla volta seguendo uno schema prefissato. Tra le euristiche disponibili, l'algoritmo implementato è \texttt{Nearest Neighbor}. Questo tipo di algoritmo ha una natura \textit{greedy}, in quanto va ad aggiungere il vertice più vicino al vertice attualmente in coda al cammino parziale.

\begin{lstlisting}[mathescape=true]
Nearest_Neighbor (V) // V = list of vertexes
	1) 'Initialization'
	   Select $v_0$ from V as starting vertex of the partial path
	2) 'Selection'
	   Given ($v_0$...$v_k$) as partial path, the next vertex $v_{k+1}$ not added yet 
	   to the partial path is the nearest one to $v_k$. 
	3) 'Insertion'
	   Insert $v_{k+1}$ right after $v_k$ in the partial path
	4) Repeat the process from the step (2) until all the vertexes have
	   been inserted
	5) Close the path inserting the starting vertex $v_0$ at the end of the
	   partial path ($v_0$...$v_n$)
\end{lstlisting}

La soluzione ritornata dall'euristica è $log(n)$-approssimata a TSP.

\subsection{Funzioni}
L'implementazione del problema non ha richiesto particolari strutture dati. Di conseguenza sono state create delle apposite funzioni per la costruzione dell'algoritmo:
\begin{itemize}
	\item \texttt{\textbf{closest\_vertex(p, V)}}: metodo che permette di calcolare la distanza minima tra il vertice \texttt{p}, il quale è si trova in coda alla lista di vertici del cammino, e la lista di vertici \texttt{V} non ancora inseriti. Una volta individuato il vertice più vicino, questo viene ritornato insieme alla distanza calcolata;
	\item \texttt{\textbf{nearest\_neighbor(V)}}: metodo che permette di costruire il circuito hamiltoniano seguendo le regole dell'euristica \texttt{Nearest Neighbor}. 
\end{itemize}

\subsection{Implementazione}
 La soluzione finale è la seguente:
\begin{itemize}
	\item Viene dichiarata una variabile locale \texttt{shortest\_distance} per memorizzare la distanza minima finale, alla quale viene assegnato il valore iniziale $\infty$;
	\item Il ciclo esterno scorre la lista di vertici \texttt{V}. In questo modo, ad ogni iterazione, il cammino iniziale viene instanziato con un vertice di partenza diverso; 
	\begin{enumerate}
		\item Viene eseguita una copia profonda \texttt{C} della lista \texttt{V}. La copia ottenuta è utilizzata per effettuare le operazioni di costruzione del cammino, mentre la lista originale rimane invariata;
		\item Inizializzazione: viene inizializzato il cammino iniziale contenente il vertice di partenza, il quale viene rimosso da \texttt{C};
		\item Viene dichiarata una variabile locale \texttt{current\_distance} per memorizzare la distanza del cammino corrente;
	\end{enumerate}
	\item A questo punto inizia il ciclo interno. Il processo di iterazione continua finché tutti i vertici di \texttt{C} sono stati visitati;
	\begin{enumerate}
		\item Selezione: avviene la chiamata alla funzione \texttt{closest\_vertex(p, V)};
		\item Inserimento: il vertice ritornato nella fase precedente viene inserito in coda al cammino parziale e rimosso da \texttt{C};
	\end{enumerate}  
	\item Una volta completato il ciclo interno, viene chiuso il ciclo inserendo nuovamente il vertice di partenza;
	\item Vengono confrontate le due variabili locali: se \texttt{current\_distance} $<$ \texttt{shortest\_distance}, allora \texttt{current\_distance} viene memorizzata come la soluzione finale provvisoria.
\end{itemize}

\subsection{Complessità}
L'algoritmo \texttt{Nearest Neighbor} può essere implementato in tempo proporzionale a $n^2$, dove $n$ è il numero di vertici del grafo.
\begin{itemize}
	\item La ricerca del vertice a distanza minima da un dato vertice ha complessità $O(n)$;
	\item La ricerca precedente avviene per tutti gli $n$ vertici. Complessità finale $O(n^2)$;
	
\end{itemize}
\pagebreak