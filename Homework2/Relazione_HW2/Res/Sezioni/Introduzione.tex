\section{Introduzione}
Questo elaborato ha lo scopo di illustrare il lavoro svolto per il secondo homework del corso \textit{Algoritmi Avanzati}.\\
L'homework ha come obiettivo quello di confrontare tra loro gli algoritmi per il problema intrattabile chiamato ``Problema del Commesso Viaggiatore'' (\textit{Traveling Salesman Problem - TSP}) definito come segue: date le coordinate $x$,$y$ di $N$ punti nel piano (i vertici), e una funzione di peso $w(u,v)$ definita per tutte le coppie di punti (gli archi), trovare il ciclo semplice di peso minimo che visita tutti gli $N$ punti (ciclo hamiltoniano). La funzione peso $w(u,v)$ è definita come la distanza Euclidea o Geografica tra i punti $u$ e $v$. La soluzione ottimale consiste nell'andare a calcolare il cammino di lunghezza minima.  
\\
L'homework ha come obiettivo quello di confrontare tra loro gli algoritmi esatti e con algoritmi di approssimazione:
\begin{enumerate}
	\item \textit{Algoritmo Held-Karp} (Sez.~\ref{held_karp})
	\item \textit{Euristica costruttiva} (Sez.~\ref{constructive_heuristic})
	\item \textit{Algoritmo 2-approssimato} (Sez.~\ref{two_approx})
\end{enumerate}


Il linguaggio in cui sono stati implementati questi algoritmi è \texttt{Python}.
Abbiamo deciso di utilizzare questo al contrario di altri linguaggi come \texttt{C++} o \texttt{Java} in quanto questa scelta ci ha permesso di utilizzare i \textit{Jupyter Notebook} e di programmare utilizzando l'IDE \textit{PyCharm} oppure con \textit{Google Colab}.

Nella Sez.~\ref{risultati} sono presenti i risultati, in forma tabellare, che confrontano le performance dei tre algoritmi, calcolati sul dataset dato, contenente 13 grafi di esempio, di dimensione compresa tra $14$ e $1000$ vertici e descritti in file \texttt{.tsp}.
Sono inoltre illustrati i grafici delle performance dei vari algoritmi.

Il lavoro è stato suddiviso equamente tra i membri del gruppo nel seguente modo:
\begin{itemize}
	\item Algoritmo Held-Karp: Lorenzo Busin
	\item Euristica costruttiva: Nicolò Tartaggia
	\item Algoritmo 2-approssimato: Ciprian Voinea
\end{itemize}

Nonostante questa suddivisione tutti i membri hanno collaborato all'implementazione degli algoritmi ed effettuato una verifica finale tramite \textit{peer review} cercando di seguire una linea di sviluppo comune.
\pagebreak

