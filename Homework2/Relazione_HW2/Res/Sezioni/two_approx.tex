\section{Algoritmo 2-approssimato}\label{two_approx}

\begin{lstlisting}[mathescape=true]
APPROX-TSP-TOUR(G, c)
	select a vertex r $\in$ G.V to be a root vertex
	compute a MST T for G from root r using MST-PRIM(G, c, r)
	let H be a list of vertices, ordered according to when they are first visited in a preordered tree walk of T
	return the hamiltonian cycle H
\end{lstlisting}	

L'algoritmo di 2-approssimazione utilizza come sottoprocedura l'algoritmo di Prim, che fornisce un MST il cui peso dà un lower bound sulla lunghezza del ciclo della soluzione ottima del TSP sul grafo dato.\\
Questo algoritmo sfrutta la disuguaglianza triangolare, ovvero dati tre nodi A, B, C, la distanza tra A e C può al massimo la distanza tra A e B sommata alla distanza tra B e C.\\
Fintanto che la funzione del costo soddisfa tale regola, viene utilizzato il minimum spanning tree per creare un ciclo con un costo che è massimo due volte il costo della soluzione ottima, per questo si dice 2-approssimato.

\subsection{Strutture dati}

	Per implementare l'algoritmo di 2-approssimazione è stato necessario l'utilizzo dell'algoritmo di Prim, che richiede la struttura dati \texttt{Heap}.
	Questa è stata presa dalla soluzione del primo homework e adattata in maniera da poter essere utilizzata con gli indici presenti nel dataset e con grafi completi.\\
	Sono state riutilizzate anche le strutture dati \texttt{Graph} e \texttt{Node}, adattando la prima per la costruzione di grafi completi.
	
\subsection{Funzioni}
	
	Le funzioni ausiliarie utilizzate per calcolare la soluzione dell'algoritmo di 2-approssimazione per TSP sono le seguenti:
	\begin{itemize}
		\item \texttt{\textbf{MSTPrim}}: metodo che ritorna l'MST di un grafo in input partendo dal nodo dato; 
		\item \texttt{\textbf{preorder}}: presi in input la mappa dei successori e il vertice di partenza, ritorna una lista della visita in profondità dell'albero a partire da tale nodo; 
		\item \texttt{\textbf{mst\_to\_tree}}: data in input la mappa dei predecessori ciascun nodo, la converte in un albero; 
		\item \texttt{\textbf{ham\_circ\_weight}}: metodo che, dato in input un ciclo hamiltoniano, ne calcola il peso;
		\item \texttt{\textbf{two\_approximation}}: metodo principale che implementa le operazioni principali dell'algoritmo di 2-approssimazione.
	\end{itemize}

\subsection{Implementazione}
	
	La soluzione del problema è stata implementata nel seguente modo:
	\begin{itemize}
		\item viene costruito il grafo \texttt{g} a partire dai lista di vertici letta dal file in input;
		\item si utilizza l'algoritmo di Prim per calcolare l'MST del grafo appena creato;
		\item tale MST viene convertito, tramite l'uso della funzione \texttt{preorder}, in una lista che rappresenta i padri di ciascun nodo;
		\item andando a riordinare questa lista tramite la funzione \texttt{preorder}, ottenendo quindi un ciclo hamiltoniano al quale, per chiudere il ciclo, viene appeso in coda il primo nodo, con indice 0;
		\item viene quindi calcolata la soluzione finale andando a sommare i pesi degli archi che collegano i nodi del ciclo.
	\end{itemize}
		
\subsection{Complessità}

	Per calcolare la complessità di questo algoritmo bisogna tenere in considerazione anche le operazioni che vengono svolte da Prim e, di conseguenza, quelle svolte sullo heap.\\
	La complessità totale è quindi $O(|V|^2)$, dove V indica la lista di vertici del grafo.

\pagebreak