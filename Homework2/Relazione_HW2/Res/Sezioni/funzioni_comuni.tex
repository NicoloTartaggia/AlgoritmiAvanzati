\section{Funzioni comuni}
Per questo homework sono state utilizzate delle funzioni comuni per le lettura e la gestione dell'input e per il calcolo della distanza tra due vertici.

\begin{itemize}
	\item \texttt{\textbf{import\_dataset(path)}}: metodo che permette di leggere il contenuto dell'input, utilizzando il percorso specificato dal parametro \texttt{path}. Esso memorizza e ritorna il numero totale di vertici, il formato dei vertici e la lista di vertici. I valori ritornati vengono salvati in variabili globali, in modo da essere accessibili il qualunque momento e da ridurre il numero di parametri passati alle varie funzioni; 
	\item \texttt{\textbf{weight(u, v)}}: metodo che permette di calcolare la distanza tra due vertici in base al loro formato. Se questo è di tipo \emph{EUC-2D} viene calcolata la distanza euclidea mentre se è di tipo \emph{GEO} le coordinate vengono prima convertite in radianti e, successivamente, viene calcolata la distanza geografica. 
\end{itemize}

\pagebreak
