\section{Algoritmo di Karger}\label{karger}
\begin{lstlisting}[mathescape=true]
KARGER(G, k):

	min = +$\infty$
	for i = 1 to k:		
		t = FULL_CONTRACTION(G)		
		if t < min:
			t = min
			
	return min

FULL_CONTRACTION(G):

	for i = 1 to |V|-2:
		pick a random edge e = (u, v)
		merge u and v in a single node
		remove edges between u and v
		
	return |E| // the edges between the two remaining vertices (u* ,v*)
\end{lstlisting}	

L'algoritmo di \textit{Karger} è un algoritmo randomizzato per la computazione del minimum cut di un \textit{multi-grafo} connesso.\\
L'idea su cui si basa è quella di contrarre i vertici all'interno del multi-grafo fino a quando non ne restano solamente due, i quali rappresentano gli oppposti del taglio.

Un multi-grafo è ...

Essendo che è un \textit{algoritmo randomizzato}, ovvero ..., la procedura di contrazione va eseguito k volte, con k calcolato ...

Possiamo quindi dire che l'algoritmo funziona con probabilità bassissima, che può però essere amplificata ripetendo questo processo molte volte.


\subsection{Strutture dati}

	A differenza degli altri homework, per migliorare le performance dell'algoritmo in questione, abbiamo deciso di utilizzare solamente la struttura dati \texttt{Graph}, implementata con i seguenti campi e metodi:
	\begin{itemize}
		\item \texttt{\textbf{n\_nodes}}: numero di nodi del grafo;
		\item \texttt{\textbf{n\_edges}}: numero di lati del grafo;
		\item \texttt{\textbf{nodes}}: dizionario con i nodi del grafo e le rispettive liste di adiacenza;
		\item \texttt{\textbf{edges}}: lista con i lati del grafo;
		\item \texttt{\textbf{\_\_init\_\_()}}: metodo di inizializzazione che imposta i precedenti campi a 0 o vuoti;
		\item \texttt{\textbf{addEdge(v1, v2)}}: metodo che dati due nodi aggiunge il lato che li unisce alla lista \texttt{edges};
		\item \texttt{\textbf{buildGraph(data)}} metodo che riceve in input una matrice di adiacenza e, in base a questa, ne costruisce un grafo.
	\end{itemize}
	
\subsection{Funzioni}
	
	Le funzioni ausiliarie utilizzate per calcolare la soluzione dell'algoritmo di Karger sono le seguenti:
	\begin{itemize}
		\item \texttt{\textbf{readInput(path)}}: metodo che, dato in input il path del dataset, ne legge il contenuto e restituisce la matrice di adiacenza del grafo;
		\item \texttt{\textbf{graphCopy(graph)}}: metodo che riceve in input un oggetto di tipo \texttt{Graph} e ne restituisce una copia;
		\item \texttt{\textbf{execute\_alg(path)}}: metodo che, dato in input il path del dataset, esegue la chiamata a \texttt{Karger(G)} calcolando i tempi richiesti nella consegna e confrontando i risultati ottenuti con quelli attesi.
	\end{itemize}

\subsection{Implementazione}
	
	La soluzione del problema è stata implementata nel seguente modo:
	\begin{itemize}
		\item a partire dalla lista di vertici nel file di input, viene costruito il grafo \texttt{graph};
		\item viene calcolato il valore \texttt{k} tramite la formula \texttt{int((graph.n\_nodes**2 / 2) * \\
			math.log(graph.n\_nodes))}. Se \texttt{k} supera $10000$ allora il suo valore lo fissiamo a tale soglia;
		\item per migliorare le performance della soluzione proposta, abbiamo deciso di creare in anticipo le copie dei grafi per ciascuna delle \texttt{k} volte che verrà eseguito \texttt{fullContraction(G)};
		\item viene effettuata la chiamata a \texttt{Karger(G,k)}, nel quale per \texttt{k} volte viene chiamato \texttt{fullContraction(G)}. I risultati di ciascuna chiamata vengono confrontati e viene salvato quello minimo, inoltre vengono registrati i tempi che sono presentati in formato tabellare nella Sez.~\ref{risultati}. Se il tempo di esecuzione supera quello massimo fissato, allora viene tornata la soluzione trovata fino a quel momento;
		\item per ciascuna chiamata a \texttt{fullContraction(G)}, questo metodo sceglie in maniera randomica un lato presente all'interno del grafo passato in input e unisce i vertici che ne rappresentano gli estremi in un singolo vertice. Per ciascuno degli altri vertici presenti all'interno del grafo si modificano le liste di adiacenza e aggiornando i riferimenti. Vengono inoltre diminuiti i contatori \texttt{n\_nodes} e \texttt{n\_edges} all'interno della struttura \texttt{Graph}. Questo metodo ritorna il numero rimanente di lati all'interno del grafo dopo aver eliminato $|V|-2$ nodi;
		\item i risultati vengono messi in coda alla lista e questo procedimento viene ripetuto per i rimanenti file in input non ancora valutati.
	\end{itemize}
		
\subsection{Complessità}

	Per calcolare la complessità di questo algoritmo bisogna tenere in considerazione:
	\begin{itemize}
		\item la complessità di Karger
		\item la complessità di fullContraction
	\end{itemize}
	La complessità totale dell'intero algoritmo risulta quindi essere

\pagebreak