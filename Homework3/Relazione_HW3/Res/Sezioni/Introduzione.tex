\section{Introduzione}
Questo elaborato ha lo scopo di illustrare il lavoro svolto per il terzo homework del corso \textit{Algoritmi Avanzati}.\\
L'homework ha come obiettivo quello di valutare le prestazioni dell'\textit{Algoritmo di Karger} per il problema del \textit{minimum cut} rispetto a quattro parametri:
\begin{enumerate}
	\item Il tempo impiegato dalla procedura di \textit{Full Contraction};
	\item Il tempo impiegato dall'algoritmo completo per ripetere la contrazione un numero sufficientemente alto di volte;
	\item Il \textit{discovery time}, ossia il momento in cui l'algoritmo trova per la prima volta il taglio di costo mimimo;
	\item L'errore nella soluzione trovata rispetto al risultato ottimo.
\end{enumerate}
%confrontare tra loro gli algoritmi per il problema intrattabile chiamato ``Problema del Commesso Viaggiatore'' (\textit{Traveling Salesman Problem - TSP}) definito come segue: date le coordinate $x$,$y$ di $N$ punti nel piano (i vertici), e una funzione di peso $w(u,v)$ definita per tutte le coppie di punti (gli archi), trovare il ciclo semplice di peso minimo che visita tutti gli $N$ punti (ciclo hamiltoniano). La funzione peso $w(u,v)$ è definita come la distanza Euclidea o Geografica tra i punti $u$ e $v$. La soluzione ottimale consiste nell'andare a calcolare il cammino di lunghezza minima.\\
%
%L'homework ha come obiettivo quello di confrontare tra loro gli algoritmi esatti e con algoritmi di approssimazione:
%\begin{enumerate}
%	\item \textit{Algoritmo Held-Karp} (Sez.~\ref{held_karp})
%	\item \textit{Euristica costruttiva} (Sez.~\ref{constructive_heuristic})
%	\item \textit{Algoritmo 2-approssimato} (Sez.~\ref{two_approx})
%\end{enumerate}
Come per i precedenti homework, il linguaggio in cui sono stati implementati questi algoritmi è \texttt{Python}.
Abbiamo deciso di utilizzare questo al contrario di altri linguaggi come \texttt{C++} o \texttt{Java} in quanto questa scelta ci ha permesso di utilizzare i \textit{Jupyter Notebook} e di programmare utilizzando l'IDE \textit{PyCharm} oppure con \textit{Google Colab}.\\
Nella Sez.~\ref{risultati} sono presenti i risultati, in forma tabellare, che confrontano le performance richieste dalla consegna, calcolati sul dataset dato, contenente 40 grafi di esempio, di dimensione compresa tra $6$ e $200$ vertici e descritti in file \texttt{.txt}.
Sono inoltre illustrati i grafici delle performance.\\
Il lavoro è stato suddiviso equamente tra i membri del gruppo collaborando all'implementazione dell'algoritmo ed effettuando una verifica finale del codice e dei risultati tramite \textit{peer review} cercando di seguire una linea di sviluppo comune.
\pagebreak

