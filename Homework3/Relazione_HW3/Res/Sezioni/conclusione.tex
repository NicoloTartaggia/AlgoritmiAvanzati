\section{Conclusione}

% Commentate i risultati che avete ottenuto: come si comportano gli algoritmi rispetti alle varie istanze? C'è un algoritmo che riesce sempre a fare meglio degli altri rispetto all'errore di approssimazione? Quale dei tre algoritmi che avete implementato è più efficiente?

Per quanto riguarda la prima e la seconda istanza, l'algoritmo di Held-Karp riesce a terminare entro il limite dei 90 secondi, trovando la soluzione ottima. 
Per quanto riguarda le altre istanze, sicuramente Held-Karp troverebbe la soluzione ottima se avessimo a disposizione un tempo infinito. 
Per le altre istanze, Nearest Neighbor risulta migliore: Held-Karp, allo scadere dei 90 secondi, ha una soluzione ottima solamente per una parte dei sottoproblemi, proseguendo poi in maniera non più ottimale. \\
Non c'è dunque un algoritmo che fa meglio degli altri in tutte le istanze dei problemi proposti, dato un limite di tempo all'algoritmo di Held-Karp.\\
Tuttavia, Nearest Neighbor è quello che presenta mediamente tempi di esecuzione ed errori inferiori rispetto agli altri algoritmi.\\
\`E stato scelto di impiegare l'algoritmo di Prim in quanto presenta una complessità minore, $O(|V|^2)$, rispetto a quello di Kruskal.
%Tra gli algoritmi di Kruskal e di Prim per il problema dell'albero di copertura minimo, è stato scelto quello di Prim poiché, essendo il grafo completo ed utilizzando l'implementazione ``ingenua'' con un vettore come coda di priorità, si ottiene una minore complessità (ovvero $O(|V|^2)$). 
Anche l'euristica Nearest Neighbor ha complessità $O(|V|^2)$, ma dai tempi rilevati si può dire che quest'ultima ha delle costanti asintotiche più basse.\\
Naturalmente l'algoritmo meno efficiente è Held-Karp che ha complessità $O(2^{|V|}*|V|^2)$, che però garantisce una soluzione ottima quando termina, ovvero avendo un tempo sufficiente.

\pagebreak