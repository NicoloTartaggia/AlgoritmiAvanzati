\section{Algoritmo di Kruskal con implementazione "naive"}
\begin{lstlisting}[mathescape=true]
KRUSKAL (G)
A = $\emptyset$
sort edges of G by cost
for each edge e, in non decreasing order of cost do
	if A U {e} is acyclic then
		A = A U {e}
return A	
\end{lstlisting}
Kruskal "naive" è un algoritmo greedy che non utilizza particolari strutture dati e che costruisce un MST aggiungendo ad ogni iterazione un nuovo lato di costo minimo all'insieme A se ciò non comporta il verificarsi di un ciclo.
\subsection{Strutture dati}
\subsubsection{Graph}
L'oggetto contiene le informazioni relative al grafo in esame:
\begin{itemize}
	\item \textbf{n}: campo dati che indica numero di vertici;
	\item \textbf{Le}: campo dati che contiene la lista degli archi;
	\item \textbf{adj}: campo dati che contiene le liste di adiacenza tramite un dizionario;
	\item \textbf{addEdge(u, v)}: metodo che dati due vertici in input ne aggiunge il collegamento nelle liste di adiacenza di entrambi i vertici;
	\item \textbf{isConnected(s, t)}: metodo che dati due vertici in input ritorna \texttt{"True"} se essi sono connessi o \texttt{"False"} se non lo sono;
	\item \textbf{buildGraph(input)}: metodo che dato in input il file che contiene la struttura di un grafo suddiviso per linee costruisce un oggetto \texttt{Graph()}.
\end{itemize} 
\subsubsection{Edge}
L'oggetto edge contiene le informazioni relative ad un arco:
\begin{itemize}
	\item \textbf{v1}: vertice di partenza;
	\item \textbf{v2}: vertice di arrivo;
	\item \textbf{weight}: peso dell'arco.
\end{itemize}
\textbf{N.B.} Non vi è una vera e propria distinzione tra vertice di partenza e vertice di arrivo dato che il grafo non è orientato.

\subsection{Implementazione}
La soluzione del problema è stata implementata nel seguente modo:
\begin{itemize}
	\item Viene inizializzato il grafo G attraverso il costruttore \texttt{Graph()} al quale viene passato in input il numero di vertici estratti dalla prima riga del file di input .txt;
	\item Viene costruito il grafo G tramite il metodo \texttt{buildGraph()}, al quale viene passato il grafo di input come parametro. La funzione, per ogni tripla (vertice1\_arco\_i, vertice2\_arco\_i, peso\_arco\_i) dell'input, aggiorna le liste di adiacenza di entrambi i vertici attraverso il metodo \texttt{addEdge()} e inserisce l'oggetto arco nella lista degli archi; 
	\item Una volta creato il grafo, esso viene fornito come input all'algoritmo \texttt{Kruskal} che esegue i seguenti passi:
	\begin{enumerate}
		\item Viene inizializzato il grafo A con lo stesso numero di vertici del grafo di input. Esso ha lo scopo di tenere traccia degli archi aggiunti al MST;
		\item Viene ordinata la lista contenente gli archi del grafo di input in maniera crescente in base al peso attraverso l'algoritmo MergeSort;
		\item Per ogni arco viene verificato, attraverso il metodo isConnected della classe Graph, che la sua aggiunta al grafo A non porti alla creazione di un ciclo all'interno del medesimo grafo e in caso positivo l'arco viene aggiunto all'insieme A. In questo caso l'algoritmo non verifica che non ci sia un ciclo, ma verifica che non ci sia già un cammino che collega i due vertici in input, perché se così fosse allora l'aggiunta di un ulteriore cammino andrebbe sicuramente a costituire un ciclo in quel grafo. Il metodo isConnected è stato implementato con una variante iterativa di una BFS che dati in input due nodi \emph{s} e \emph{t} inizialmente considera tutti i vertici, a parte quello di partenza \emph{s}, come non visitati e tramite una coda vengono man mano visitati i nodi adiacenti fino ad arrivare al nodo \emph{t} se questo è raggiungibile da \emph{s}. Per ottimizzare l'algoritmo è possibile fermare il ciclo quando A ha \emph{n-1} lati, dove \emph{n} indica il numero di vertici.
	\end{enumerate}
\end{itemize}

\subsection{Complessità}
La complessità dell'intero algoritmo è \textbf{O(mn)} e si ottiene considerando l'algoritmo impiegato per l'ordinamento degli archi in base al peso, che in questo caso è il MergeSort e quindi ha complessità \textbf{O(m log(m))} con \emph{m} che indica il numero di archi, ed il ciclo for che effettua al più \emph{m} iterazioni, in cui per ognuna si valuta se il lato in esame può essere aggiunto alla soluzione: in questo caso il controllo sfrutta una variante della BST e quindi ha complessità lineare nel numero di vertici \textbf{O(n)}.

\pagebreak