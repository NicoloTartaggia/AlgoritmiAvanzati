\section{Algoritmo di Kruskal con Union-Find}

\begin{lstlisting}[mathescape=true]
KRUSKAL(G):
	A = $\emptyset$
	For each vertex v $\in$ G.V:
		MAKE-SET(v)
	For each edge (u, v) $\in$ G.E ordered by increasing order by weight(u, v):
		if FIND-SET(u) $\neq$ FIND-SET(v):       
			A = A $\cup$ {(u, v)}
			UNION(u, v)
	return A
\end{lstlisting}

Come per la sua versione naive, l'implementazione di Kruskal tramite l'uso della struttura dati Union Find è un algoritmo greedy che costruisce un MST aggiungendo ad ogni iterazione un nuovo lato di costo minimo all'insieme A se i vertici di questo non fanno già parte dello stesso set.

\subsection{Strutture dati}

	\subsubsection{Graph ed Edge}
		L'implementazione di queste due strutture dati sono equivalenti a quelle utilizzate in \ref{kruskal_naive_graph} e \ref{kruskal_naive_edge} per l'algoritmo di Kruskal naive.
	
	\subsubsection{Union Find}
		La struttura dati \textit{Union Find} (o \textit{Disjoint-Set}) mantiene una collezione di insiemi dinamici disgiunti (chiamati anche \textit{componenti connesse}) $S = \{s_1, s_2, ... , s_n\}$.
		Viene implementata tramite un array e visualizzabile come un insieme di alberi diretti, in quanto c'è una relazione padre-figlio dove ciascun nodo figlio punta al proprio padre.\\
		Per ogni elemento di questo array viene salvato un campo parent che, al momento dell'inizializzazione, contiene l'indice dell'array dell'oggetto x.\\
		Per ciascun insieme dell'oggetto Union Find c'è un elemento che lo rappresenta, detto anche \textit{root}. 
		La profondità di questi insiemi è pari al numero di elementi compresi tra l'ultimo nodo figlio e la root.\\
		Le operazioni permesse sugli elementi di questa struttura dati sono:
		\begin{itemize}
			\item \texttt{\textbf{Initialize(x)}} o \texttt{\textbf{Make-Set(x)}}: metodo che dato un insieme di oggetti $X = \{x_1, x_2, ... , x_n\}$ crea una struttura dati Union Find nella ciascun elemento di $x_i \in X$ viene creato un set con root $x_i$.
			Questa operazione ha complessità lineare sul numero di elementi in $X$, quindi $O(n)$;
			\item \texttt{\textbf{Find(x)}}: metodo che ritorna la radice dell'insieme in cui è contenuto l'elemento $x$ passato in input, risalendo di padre in padre fino a trovare un \textit{self loop}, ovvero trova un elemento che punta a se stesso. 
			La sua complessità è proporzionale alla profondità della radice di $x$, quindi nel caso peggiore è $O(n)$;
			\item \texttt{\textbf{Union(x, y)}}: metodo che prende in input due oggetti e, se questi fanno parte di due insiemi disgiunti diversi, unisce le due componenti connesse che li contengono facendo puntare la radice dell'albero con profondità minore a quella del secondo albero, minimizzando quindi la profondità totale. \\
			La complessità di questa operazione è $O(log n)$ e viene dominata dalla complessità della precedente funzione \texttt{\textbf{Find}}.
		\end{itemize}

	\subsection{Implementazione}
		La soluzione al problema è stata implementata nel seguente modo:
		\begin{itemize}
			\item viene inizializzato il grafo $G$ attraverso il costruttore \texttt{Graph()};
			\item viene costruito il grafo $G$ tramite il metodo \texttt{buildGraph()}, al quale viene passato il grafo di input come parametro.
			La funzione, per ogni tripla ($vertice_1\_arco_i$, $vertice_2\_arco_i$, $peso\_arco_i$) dell'input, aggiorna le liste di adiacenza di entrambi i vertici attraverso il metodo \texttt{addEdge()} e inserisce l'oggetto arco nella lista degli archi; 
			\item una volta creato il grafo, esso viene fornito come input all'algoritmo \texttt{KruskalUF} che esegue i seguenti passi:
			\begin{enumerate}  				
				\item viene definito $A$ un insieme vuoto, a questo vengono aggiunti ad ogni iterazione i lati che andranno a costituire la soluzione;
				\item viene inizializzata la struttura di insiemi disgiunti $U$ passando l'insieme di nodi del grafo al suo costruttore, questo crea un insieme per ciascun nodo del grafo;
				\item utilizzando l'algoritmo MergeSort, viene ordinata la lista contenente gli archi del grafo di input in maniera crescente;		
				\item viene iterata la lista di lati ordinata e per ciascuno di questi viene controllato tramite se gli insiemi dei suoi nodi coincidono confrondandone la root, se queste sono diverse significa i nodi sono in due componenti connesse distinte, in tal caso viene aggiungo il lato al grafo di soluzione $A$ aggiornando la situazione delle componenti connesse.
			\end{enumerate}
		\end{itemize}
	
	\subsection{Complessità}
		Per calcolare la complessità totale dell'algoritmo bisogna tenere in conto:
		\begin{itemize}
			\item il costo $O(n)$ dell'inizializzazione della struttura dati Union Find;
			\item il MergeSort con complessità $O(m log(n))$, dove $m$ indica il numero di archi;
			\item le operazioni del ciclo, effettuate al massimo $m$ volte:
			%TODO Verificare complessità correttamente
			\begin{itemize}
				\item le due operazioni di Find ($2m$ in totale) con complessità $O(m \log n)$
				\item le $n-1$ operazioni di union (eseguita ogni volta che viene aggiunto un nuovo lato) con complessità $O(n \log n)$
				\item l'aggiornamento di $A$ ha un costo lineare, quindi con complessità $O(m)$
			\end{itemize}
		\end{itemize}
		La complessità dell'intero algoritmo è quindi $O(mlogn)$.

\pagebreak